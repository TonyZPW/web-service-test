\documentclass[12pt]{beamer}

\usetheme{Air}
\usepackage[czech]{babel}
\usepackage{thumbpdf}
\usepackage{wasysym}
\usepackage{ucs}
\usepackage[utf8]{inputenc}
\usepackage{verbatim}

\pdfinfo
{
  /Title       (Web Services)
  /Creator     (TeX)
  /Author      (Matěj Laitl, Matěj Novotný)
}


\title{Web Services}
\subtitle{Aby byly strojově čitelné zprávy čitelné i pro vás}
\author{Matěj Laitl, Matěj Novotný}
\date{\today}

\begin{document}

\frame{\titlepage}

\section*{}
\begin{frame}
  \frametitle{Co si dnes povíme}
  \tableofcontents[section=1,hidesubsections]
\end{frame}

\AtBeginSection[]
{
  \frame<handout:0>
  {
    \frametitle{Přehled}
    \tableofcontents[currentsection,hideallsubsections]
  }
}

\AtBeginSubsection[]
{
  \frame<handout:0>
  {
    \frametitle{Přehled}
    \tableofcontents[sectionstyle=show/hide,subsectionstyle=show/shaded/hide]
  }
}

\newcommand<>{\highlighton}[1]{%
  \alt#2{\structure{#1}}{{#1}}
}

\newcommand{\icon}[1]{\pgfimage[height=1em]{#1}}



%%%%%%%%%%%%%%%%%%%%%%%%%%%%%%%%%%%%%%%%%
%%%%%%%%%% Content starts here %%%%%%%%%%
%%%%%%%%%%%%%%%%%%%%%%%%%%%%%%%%%%%%%%%%%



\section{Web Services}

\begin{frame}
  \frametitle{Co jsou Web Services}
  \framesubtitle{...aby nebyl nikdo zmatený}

  \begin{alertblock}{Web Services nejsou:}
    \begin{itemize}
      \item Prohlížení webu skrz webový prohlížeč
      \item Na web zaměřené služby typu webhosting
    \end{itemize}
  \end{alertblock}
\end{frame}

\begin{frame}
  \frametitle{Co jsou Web Services}
  \framesubtitle{Definice - Věta - Důkaz}

  \uv{Mechanismus pro výměnu strojově čitelných dat po síti}

  \vspace{5mm}

  \begin{block}{Web Services jsou:}
    \begin{itemize}
      \item nezávislé na programovacím jazyku
      \item nezávislé na architektuře zůčastněných strojů
      \item uzpůsobené k překonání NATů a FireWallů
    \end{itemize}
  \end{block}
\end{frame}

\begin{frame}
  \frametitle{Vlastnosti Web Services}

  \begin{block}{Klíčové vlastnosti:}
    \begin{itemize}
      \item výměna probíhá mezi \highlighton{klientem} a \highlighton{serverem}
      \item data jsou \highlighton{strukturovaná}
      \item rozhraní je popsané ve \highlighton{strojově zpracovatelném tvaru}
      \item pro komunikaci se využívá \highlighton{aplikační vrstva} síťového modelu (\highlighton{ http} nebo \highlighton{https})
    \end{itemize}
  \end{block}
\end{frame}

\begin{frame}
  \frametitle{Rozdělení Web Services}
  \framesubtitle{...abychom si to zaškatulkovali}
	\begin{itemize}
      \item \LARGE Big web services
		\begin{itemize}
			\item Podniková řešení
			\item SOAP, XML, WSDL
		\end{itemize}
      \item \LARGE RESTful
		\begin{itemize}
			\item API webových služeb
		\end{itemize}
    \end{itemize}
\end{frame}

\begin{frame}
	\frametitle{Big web services}
	\begin{itemize}
		\item definice rozhraní pomocí \highlighton{WSDL} (Web Services Description Language)
		\item odeslání požadavku pomocí \highlighton{http}
		\item formát požadavku i odpovědi specifikuje \highlighton{SOAP} (Simple Object Access Protocol)
	\end{itemize}
\end{frame}

\begin{frame}
  \frametitle{RESTful Web Services}
  \framesubtitle{}

  Tady Matěj napíše něco o RESTful WS. Možná na to spotřebuje i 2 slidy
\end{frame}

\begin{frame}
  \begin{example}
    Od teď dále se budeme věnovat tzv. Big Web Services
  \end{example}
\end{frame}

\section{SOAP}

\begin{frame}
  \frametitle{Něco málo o SOAPu}
  \framesubtitle{...nikoliv o mýdle}

  Dříve \uv{Simple Object Access Protocol}, nyní prostě \uv{SOAP}

  \begin{block}{SOAP:}
    \begin{itemize}
      \item je protokol upravující
    \end{itemize}
  \end{block}
\end{frame}

\section{WSDL}

\begin{frame}
  \frametitle{Přístupy k implementaci serveru}
  \framesubtitle{Na co si dát pozor}

  \begin{itemize}
    \item Napsat WSDL, vygenerovat prázdnou implementaci
    \begin{itemize}
      \item \alert{Pozor na na pozdější úpravy}
    \end{itemize}
    \item Naimplementovat server, vygenerovat WSDL
    \item Napsat obojí ručně
    \begin{itemize}
      \item Pro otrlé
    \end{itemize}
  \end{itemize}
\end{frame}

\section{Ukázky}

\begin{frame}
  \frametitle{Generovaní Objective-C klienta z WSDL}
  \framesubtitle{...bylo by nudné to psát ručně}

  \begin{example}
    Matěji, a teď ukaž, jak se generuje klient v Obj-C a jak to pak vypadá
  \end{example}
\end{frame}

\begin{frame}
  \frametitle{Dynamicky vytvořený klient v Pythonu}
  \framesubtitle{...někdo generování nepotřebuje}

  \begin{example}
    Matěji, a teď ukaž, jak to lze celé udělat na 3 řádky
  \end{example}
\end{frame}

\begin{frame}
  \frametitle{Server v Pythonu}
  \framesubtitle{...čím línější programátor, tým vyšší jazyk}

  \begin{example}
    Matěji, a teď ukaž, jak udělat server v Pythonu
  \end{example}
\end{frame}

\begin{frame}
  \frametitle{SOAP zpráva}
  \framesubtitle{...pohled pod pokličku}

  \begin{example}
    Matěji, teď se kouknem na SOAP zprávu zachycenou nebo někde zobrazenou
  \end{example}
\end{frame}


%%%%%%%%%%%%%%%%%% PUVODNI KOD NIZE %%%%%%%%%%%%%%%%%%%

\section{Fancy features}
\begin{frame}
  \frametitle{Highlighting}
  \framesubtitle{Hey! Look here! here!}

  \begin{block}{A regular block}
  \begin{itemize}
    \item Normal text
    \item \highlighton{Highlighted text} to draw attention
    \item \alert{"Alert'ed" text} to spot very important information
    \item Alternatively you can
    \begin{itemize}
      \alert{\item "Alert" the item itself}
      \highlighton{\item Or "Highlight" it}
    \end{itemize}
  \end{itemize}
  \end{block}
  \begin{alertblock}{If it's very very important...}
  \alert{... you can "alert" in an "alertblock"}\\
  Ewww, nasty, heh?
  \end{alertblock}
\end{frame}

%%%%%%%%%%%%%%%%%%%%%%%%%%% PUVODNI KOD VYSE %%%%%%%%%%%%%%%%%%%%

\begin{frame}
  \frametitle{Zdroje}
  \framesubtitle{Pokud se chcete dovědět víc}
  \begin{thebibliography}{10}

  \beamertemplatearticlebibitems

  \bibitem{projekt-na-githubu}
    Zdrojáky serveru, klientů a prezentace
    \newblock {\tt http://github.com/strohel/web-service-test}

  \end{thebibliography}
\end{frame}

\begin{frame}
  \vspace{2cm}
  {\huge Otázky?}

  \vspace{3cm}
  \begin{flushright}
    Matěj Laitl, Matěj Novotný

    \structure{\footnotesize{matej@laitl.cz, mates.novotny@gmail.com}}
  \end{flushright}
\end{frame}

\end{document}
